MQTT \+:
\begin{DoxyItemize}
\item Pour \char`\"{}\+Message Queuing Telemetry Transport\char`\"{}, est un protocole open source
\item de messagerie qui assure des communications non permanentes entre des appareils par le transport de leurs messages.
\end{DoxyItemize}

Adafruit \+:
\begin{DoxyItemize}
\item Adafruit Industries est une entreprise fondée en 2005 par Limor Fried, une ingénieure américaine,
\item qui se spécialise dans la vente et la production de composants électroniques et de matériel libre. ~\newline

\end{DoxyItemize}

ESP32 \+:
\begin{DoxyItemize}
\item ESP32 est une famille de microcontrôleurs à base de processeurs 32 bits,
\item développée par Espressif Systems, une société chinoise basée à Shanghai.
\end{DoxyItemize}

STATION (WIFI\+\_\+\+STA) \+:
\begin{DoxyItemize}
\item Le mode Station (STA) est utilisé pour connecter le module ESP32 à un point d’accès Wi-\/\+Fi.
\item L’\+ESP32 se comporte comme un ordinateur qui serait connecté à notre box. Si la box est reliée à Internet, alors l’\+ESP32 peut accéder à Internet. L’\+ESP32 peut se comporter en tant que client, c’est-\/à-\/dire faire des requêtes aux autres appareils connectés sur le réseau ou en tant que serveur , c’est-\/à-\/dire que d’autres appareils connectés sur le réseau vont envoyer des requêtes à l’\+ESP32. Dans les 2 cas, l’\+ESP32 peut accéder à Internet.
\end{DoxyItemize}

AP (Access Point) (WIFI\+\_\+\+AP) \+:
\begin{DoxyItemize}
\item En mode Access Point, l’\+ESP32 se comporte comme un réseau Wi-\/\+Fi (un peu comme une box) \+:
\item d’autres appareils peuvent s’y connecter. Dans ce mode, l’\+ESP32 n’est relié à aucun autre réseau et n’est donc pas connecté à Internet.
\item Ce mode est plus gourmand en calcul et en énergie (la carte ESP32 va chauffer) puisque l’\+ESP32 doit simuler un routeur Wifi complet (Soft AP).
\end{DoxyItemize}

MQTT \+:
\begin{DoxyItemize}
\item Pour \char`\"{}\+Message Queuing Telemetry Transport\char`\"{}, est un protocole open source
\item de messagerie qui assure des communications non permanentes entre des appareils par le transport de leurs messages.
\end{DoxyItemize}

Adafruit \+:
\begin{DoxyItemize}
\item Adafruit Industries est une entreprise fondée en 2005 par Limor Fried, une ingénieure américaine,
\item qui se spécialise dans la vente et la production de composants électroniques et de matériel libre. ~\newline

\end{DoxyItemize}

ESP32 \+:
\begin{DoxyItemize}
\item ESP32 est une famille de microcontrôleurs à base de processeurs 32 bits,
\item développée par Espressif Systems, une société chinoise basée à Shanghai.
\end{DoxyItemize}

STATION (WIFI\+\_\+\+STA ) \+:
\begin{DoxyItemize}
\item Le mode Station (STA) est utilisé pour connecter le module ESP32 à un point d’accès Wi-\/\+Fi.
\item L’\+ESP32 se comporte comme un ordinateur qui serait connecté à notre box. Si la box est reliée à Internet, alors l’\+ESP32 peut accéder à Internet. L’\+ESP32 peut se comporter en tant que client , c’est-\/à-\/dire faire des requêtes aux autres appareils connectés sur le réseau ou en tant que serveur , c’est-\/à-\/dire que d’autres appareils connectés sur le réseau vont envoyer des requêtes à l’\+ESP32. Dans les 2 cas, l’\+ESP32 peut accéder à Internet.
\end{DoxyItemize}

AP (Access Point) (WIFI\+\_\+\+AP ) \+:
\begin{DoxyItemize}
\item En mode Access Point, l’\+ESP32 se comporte comme un réseau Wi-\/\+Fi (un peu comme une box) \+:
\item d’autres appareils peuvent s’y connecter dessus. Dans ce mode, l’\+ESP32 n’est relié à aucun autre réseau et n’est donc pas connecté à Internet.
\item Ce mode est plus gourmand en calcul et en énergie (la carte ESP32 va chauffer) puisque l’\+ESP32 doit simuler un routeur Wifi complet (Soft AP). 
\end{DoxyItemize}