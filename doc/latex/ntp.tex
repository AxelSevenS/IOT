Quelle heure est il ?

Les cartes Arduino, ESP8266 et ESP32 ne disposent pas d’horloge temps réel. La seule information dont dispose la carte est le nombre

Nous allons donc récupérer l\textquotesingle{}heure auprès d\textquotesingle{}un serveur de temps NTP \+: Network Time Protocol. Network Time Protocol (« protocole de temps réseau ») est un protocole qui permet de synchroniser, via un réseau informatique, l\textquotesingle{}horloge locale d\textquotesingle{}ordinateurs sur une référence d\textquotesingle{}heure.

Nous pourrons ainsi horodater (timestamp) des mesures, connaître le temps écoulé entre deux événements, afficher l’heure courante sur une interface WEB, déclencher une action programmée ...

La bibliothèque NTPClient est nécessaire.

Fichier \mbox{\hyperlink{_my_n_t_p_8h}{My\+NTP.\+h}} 