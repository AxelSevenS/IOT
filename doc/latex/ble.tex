Dans le cadre de ce cours de découverte de l\textquotesingle{}IoT nous utiliserons les fonctionnalités de base du BLE, dans un mode potentiellement dégradé. L\textquotesingle{}idée est que vous preniez simplement la main sur le module BLE et que vous l\textquotesingle{}intégriez à l\textquotesingle{}ensemble du projet.

En mode serveur chaque équipement BLE
\begin{DoxyItemize}
\item Permet d\textquotesingle{}être vu par les autres équipements BLE,
\item Propose différents services,
\item Chaque service propose différentes caractéristiques en mode lecture ou écriture. Il existe une nomenclature des services et caractéristiques, que nous n\textquotesingle{}utiliserons pas ici.
\end{DoxyItemize}



En mode client, il est possible de scanner les équipements bluetooth et de potentiellement se connecter à eux pour échanger des données.

Les serveurs \char`\"{}publient\char`\"{} ou font la promotion de leurs services pour que les clients puissent savoir ce qu\textquotesingle{}ils proposent avant de se connecter. Dans notre cas, nous publierons simplement un service et un nom d\textquotesingle{}équipement. Cela sera suffisant pour savoir quels sont les équipements à proximité qui proposent un service donné \char`\"{}\+Contact Tracker\char`\"{} et leurs noms. 